

%----------------------------------------------------------------------------------------
%	PACKAGES AND OTHER DOCUMENT CONFIGURATIONS
%----------------------------------------------------------------------------------------

\documentclass[11pt, a4paper, oneside]{Thesis} % Paper size, default font size and one-sided paper
\usepackage{lipsum}

\usepackage[latin1]{inputenc}% per macchine Linux/Mac/UNIX/Windows; meglio utf8
\usepackage[T1]{fontenc} % To insert directly < > =
\usepackage{lmodern}

\usepackage[expert]{mathdesign}

\graphicspath{{./Pictures/}} % Specifies the directory where pictures are stored
\usepackage{graphicx} % to insert jpg pictures
\usepackage{colortbl}

%\usepackage{floatrow} % to put figure and table side by side, e.g. in Air pollution work.
% Table float box with bottom caption, box width adjusted to content
%\newfloatcommand{capbtabbox}{table}[][\FBwidth]

\usepackage{enumitem}
\setlist{parsep=0pt,listparindent=\parindent}

%-------------- from Bike sharing paper  -----------------------------------------------------------------------------
%\usepackage{subcaption}

\newcommand{\BIKEDef}{{\cal ST}ation {\cal O}ccupancy {\cal P}redictor} 
\newcommand{\BIKE}{STOP} 
%------------------------------------------------------------------------------------------------------------------------------


%-------------- from ESWA paper ----------------------------------------------------------------------------------------
\usepackage{amsmath} % for insert equation including \binom{}{}
\usepackage{multirow} % to insert multi rows inside table
\usepackage{booktabs}

\usepackage[]{algorithm2e}
\usepackage{subfigure}


\newtheorem{Definition}{Definition}[section]
%\newcommand{\MLC}{{\sc M}{\sc L}{\sc C}} 
\newcommand{\MLDA}{{\sc M}{\sc L}{\sc D}{\sc A}} 
\newcommand{\MLDAdef}{{\sc M}ultiplel-{\sc L}evel {\sc D}ata {\sc A}nalysis}
\newcommand{\MLCDef}{{\sc M}ultiplel-{\sc L}evel {\sc C}lustering}

\newcommand{\comment}[1]{}
%------------------------------------------------------------------------------------------------------------------------------

%-------------- from ICHI paper ----------------------------------------------------------------------------------------
\usepackage{multirow} % to insert multi rows inside table
%\usepackage{cite}
%------------------------------------------------------------------------------------------------------------------------------

%--------------from CBMS paper  ----------------------------------------------------------------------------------------
\newcommand{\CARPE}{{\sc CRP}}
\newcommand{\CARPEdef}{{\sc C}ardiopulmonary {\sc R}esponse {\sc P}rediction}

\newcommand{\MWL}{W$_{peak}$}
\newcommand{\MWLQ}{$\widehat{\mbox{W}}_{peak,Q}$}
\newcommand{\MWLT}{W$_{peak,\theta}$}

\newcommand{\individual}{\textit{single-test}}
\newcommand{\community}{\textit{multiple-test}}
%------------------------------------------------------------------------------------------------------------------------------

%--------------from Air Pollution paper  ----------------------------------------------------------------------------------------
\newcommand{\frameworkDef}{{\sc GE}neralized {\sc C}orrelation analyzer of p{\sc O}llution data}
\newcommand{\framework}{{\sc GECkO}} 
%------------------------------------------------------------------------------------------------------------------------------

%--------------from Spatio-Temporal twitter paper  ----------------------------------------------------------------------------------------
\newcommand{\TTTlong}{{\sc T}witter {\sc Miner}} 
\newcommand{\TTT}{{\sc TMiner}} 
%------------------------------------------------------------------------------------------------------------------------------

%--------------from Patient Flow paper  ----------------------------------------------------------------------------------------
\newcommand{\TTTlongP}{\textit{PA}tient \textit{TR}ansfer \textit{AN}alysis} 
\newcommand{\TTTP}{{\sc PATRAN}} 
%------------------------------------------------------------------------------------------------------------------------------

\usepackage{float} % to control the position of the picture with [H] added after \begin{figure}

\usepackage[square, numbers, comma, sort&compress]{natbib} % Use the natbib reference package - read up on this to edit the reference style; if you want text (e.g. Smith et al., 2012) for the in-text references (instead of numbers), remove 'numbers' 

\hypersetup{urlcolor=blue, colorlinks=true} % Colors hyperlinks in blue - change to black if annoying
\title{\ttitle} % Defines the thesis title - don't touch this

\usepackage{indentfirst} 
\usepackage{amssymb} % >= <=

\usepackage{footnote} %to insert footnotes together with \begin{savenotes}. must be put after the color packages.

%\renewcommand\bibname{References} %change Bibliography to References
\begin{document}

\frontmatter % Use roman page numbering style (i, ii, iii, iv...) for the pre-content pages

\setstretch{1.3} % Line spacing of 1.3

% Define the page headers using the FancyHdr package and set up for one-sided printing
\fancyhead{} % Clears all page headers and footers
\rhead{\thepage} % Sets the right side header to show the page number
\lhead{} % Clears the left side page header

\pagestyle{fancy} % Finally, use the "fancy" page style to implement the FancyHdr headers

\newcommand{\HRule}{\rule{\linewidth}{0.5mm}} % New command to make the lines in the title page

% PDF meta-data
\hypersetup{pdftitle={\ttitle}}
\hypersetup{pdfsubject=\subjectname}
\hypersetup{pdfauthor=\authornames}
\hypersetup{pdfkeywords=\keywordnames}

%----------------------------------------------------------------------------------------
%	TITLE PAGE
%----------------------------------------------------------------------------------------

\begin{titlepage}
\begin{center}

\textsc{\LARGE \univname}\\[1.5cm] % University name

%\large{\groupname}\\
\large \DEPTNAME\\[1.3cm] % Research group name and department name

\textsc{\Large PhD Thesis}\\[0.7cm] % Thesis type

%\HRule \\[0.4cm] % Horizontal line
{\huge \bfseries \ttitle}\\[4cm] % Thesis title
%\HRule \\[1.5cm] % Horizontal line

\begin{figure}[H]
	\centering
		\includegraphics[scale=0.3]{./Figures/Polito.jpg}
\end{figure} 


\begin{minipage}{0.4\textwidth}
\begin{flushleft} \large
\emph{Supervisor:}\\
{\supname} % Author name - remove the \href bracket to remove the link
\end{flushleft}
\end{minipage} 
\begin{minipage}{0.4\textwidth}
\begin{flushright} \large
\emph{Author:} \\
{\authornames} % Supervisor name - remove the \href bracket to remove the link  
\end{flushright}
\end{minipage}\\[3cm]

 
{\large \today} % Date
%\includegraphics{Logo} % University/department logo - uncomment to place it
 
\vfill
\end{center}

\end{titlepage}


%----------------------------------------------------------------------------------------
%	ABSTRACT PAGE
%----------------------------------------------------------------------------------------

\addtotoc{Abstract} % Add the "Abstract" page entry to the Contents

\abstract{\addtocontents{toc}{\vspace{1em}} % Add a gap in the Contents, for aesthetics



}

\clearpage % Start a new page

%----------------------------------------------------------------------------------------
%	ACKNOWLEDGEMENTS
%----------------------------------------------------------------------------------------

\acknowledgements{\addtocontents{toc}{\vspace{1em}} % Add a gap in the Contents, for aesthetics



}
\clearpage % Start a new page
%----------------------------------------------------------------------------------------
%	LIST OF CONTENTS/FIGURES/TABLES PAGES
%----------------------------------------------------------------------------------------

\pagestyle{fancy} % The page style headers have been "empty" all this time, now use the "fancy" headers as defined before to bring them back
\lhead{\emph{Contents}} % Set the left side page header to "Contents"

\setcounter{tocdepth}{2} % Set the depth of table of contents, 2 = till subsection
\tableofcontents % generate the Table of Contents automatically
\listoffigures
\listoftables

%----------------------------------------------------------------------------------------
%	THESIS CONTENT - CHAPTERS
%----------------------------------------------------------------------------------------

\mainmatter % Begin numeric (1,2,3...) page numbering

\pagestyle{fancy} % Return the page headers back to the "fancy" style

\chapter{Introduction}
In the last years, we have literally been overwhelmed with data. 
We have witnessed, at the same moment, very strong advances in the domain of data generation, data collection and data storage.
Just think about the new social applications which gathers information about every possible aspects of the users. From the voluntary data (tweets, comments, pictures) to data extracted with less straightforward techniques (cookies, pointer tracking, machine learning algorithms applied to photo repositories,...). What about the data generated by the wearable devices, or by the car black-boxes installed by car insurances on the customers' cars?
The advances related to data generation and collection came together with the possibility of storing data which we would have trashed in the past. The reason behind this new trend about gathering as much data as one can is related to the new value that is given to such data.
Everybody are collecting data because it is useful. And if it is not clear how can be exploited now, probably it will be useful in the future.
Lying hidden in all this raw data is potentially useful knowledge, which is rarely exploited. 

The value of these data is directly correlated to the knowledge which can be extracted from it. It is very related to the use cases. Therefore, for example, it is possible to think about companies which, through the analysis of huge amount of customer attributes, are able to develop predictive models which target customers. Another example could be related to the incredible amount of data collected by sensors in the automotive domain. The possible exploitation of this information are several: from self-driving car algorithms training to predictive component fixing. Finally, many efforts are nowadays spent in pre-crime projects. By means of big data and prediction models, crimes are predicted and customized counter-measures are adopted.

In a scenario characterized by this huge amount of valuable data,  the interest towards Data mining, which is a branch of computer science which extracts useful and effective knowledge from data, has risen. The trend is noticeable in both industrial and academic environments. For the companies, as already mentioned, the ability to exploit deal with big data is demonstrating to be a concrete asset. From the academic point of view, the design of big data algorithms represent a very stimulant challenge. The application of traditional data mining techniques to such large collection of data is very challenging. As the amount of data increases, the proportion of it that people is able to interpret decreases (cit. Data mining: practical Machine learning tools and techniques). For this reason, there is a concrete need of a new generation of scalable tools which, often, need to be redesigned from scratches to cope with such an extreme environments.

In this dissertation, we focus on one of the most popular data mining technique, frequent itemset mining. Frequent itemset mining is an exploratory data analysis method used to discover frequent co-occurrence among the items of a transactional dataset (attribute-value pairs). Frequent itemsets are very useful for data summarization and correlation analysis. \textbf{ADD MORE ON FIM AND Association rules and rule-based classifier and raccomandation}.
Frequent itemset extraction is a very challenging problem in the big data domain. The reason is related to the nature of the problem which requires a full knowledge of the input data. In the last years several scalable techniques have been introduced. All of them relies on different search space exploration strategy and this leads to different performances related to the use case. 
\\

\textbf{Thesis statement: }\textit{This dissertation is an effort to thoroughly analyze the current scalable frequent itemset mining tools and, eventually, try to fill in the discovered gap.}

In the final part of this Chapter, we resume this dissertation plan highlighting our research contribution.

\section{Dissertation plan and research contribution}
The main contribution of this dissertation is to deeply examine the current state of the art of frequent itemset mining algorithms and its usage. Therefore, after discovering the environment lacks and issues, try to enrich it with new solutions and algorithms. 
This target is achieved through three main steps, useful to cluster together and label the research contributions behind them:
\begin{enumerate}
\item A deep analysis of the most reliable frequent itemset mining tools for big data
\item The introduction of a new scalable frequent itemset mining algorithm
\item The contribution of frequent itemsets to big data mining frameworks for the extraction of misleadig generalized itemsets
\end{enumerate}
The remainder part of this section will briefly introduce each phase. thesis work focuses on analysis and design of data mining algorithms for big data. Specifically, it will focus on Frequent Itemset Mining, a family of techniques designed to extract the most frequent patterns from transactional datasets which can be used to highlights interesting and unknown data correlations.
After a brief introduction on data mining techniques and frequent itemset mining, the most spread distributed frameworks will be presented.


Following this preliminary analysis, a thorough review of the most affirmed solutions will be introduced, evidencing the current limitation of the academic state of the art.
Then, an innovative distributed algorithm will be presented and evaluated, demonstrating its effectiveness in the context of High-Dimensional pattern mining.
Finally, before the conclusion, other two works related to the scalable frequent itemset mining will be introduced. 




\subsection{Frequent Itemset Mining for Big Data: an experimental analysis}
Itemset mining is a well-known exploratory data mining technique used to
discover interesting correlations hidden in a data collection. Since it supports
different targeted analyses, it is profitably exploited in a wide range
of different domains, ranging from retail store informations to network traffic or biological repositories. As already mentioned, with the increasing amount of generated data, different distributed and scalable algorithms
have been developed. They have been developed exploiting the computational advantages of distributed
computing platforms, such as Apache Hadoop and Apache Spark. However, depending on the use case, it is not easy to select the best fitting algorithm. Several features affects this choice, such as data cardinality or data distribution. Therefore, the algorithm selection often relies on analyst expertise. 
For this reason, the delivered analysis will examine both theoretically (survey bigdap) and experimentally (survey itemset) some state-of-the-art implementations of frequent itemset mining algorithms. The ratio is to guide the analyst in selecting the most suitable approach based on the use case and the outline lesson learned. 
The review takes into account also some aspects typical of distributed environment, such as communication costs and load balancing. Many real and synthetic datasets have been considered in the comparison.

The takeaways of the review is that no algorithm is universally superior and performances are heavily skewed by data distribution and input parameter settings. In addition, no algorithm has been designed purposely to be able to cope with high-dimensional data, and, as shown in the next subsection, we have tried to fill in this gap.

\subsection{A Parallel Map-Reduce Algorithm to Efficiently
Support Itemset Mining on High Dimensional Data}
In today's world, many scientific applications such as bioinformatics and networking, are continuously generated large volumes of data. Since
each monitored event is usually characterized by a variety of features, high-
dimensional datasets have been continuously generated. 
Frequent itemset is one of the technique used to extract value from these complex collections, discovering hidden and non-trivial correlations among data. 
Thanks to the spread of distributed and parallel frameworks, the development of scal
able approaches able to deal with the so called Big Data has been extended
to frequent itemset mining. Unfortunately, as mentioned in the previous Subsection (and clearly shown in Chapter \ref{survey}), most of the current algorithms are designed to cope with low-dimensional datasets, delivering poor performances in those use cases characterized by high-dimensional data. This work introduces 
PaMPa-HD, a MapReduce-based frequent closed itemset mining
algorithm for high dimensional datasets. An efficient solution has been pro-
posed to parallelize and speed up the mining process. Furthermore, different
strategies have been proposed to easily tune-up the algorithm parameters.
The experimental results, performed on real-life high-dimensional use cases,
show the efficiency of the proposed approach in terms of execution time, load
balancing and robustness to memory issues.

\subsection{Big Data Mining frameworks and Misleading Generalized Itemsets}
Data analysis is very large family of processes; frequent itemset mining represents just one of the steps required to deal with data. Along with other data mining algorithms, they represent just the knowledge extraction and exploration step of the whole process, which is strongly composed of many data preparation phases.
The availability of distributed and parallel platforms has allowed the design of big data mining systems. These systems, with a design which is parallel starting from the very first data preparation steps, are able to deal with the so called data revolution.

In these environments, distibuted frequent itemset mining is just one of the possible 'modules' of the framework. It can be replaced by other data mining analyses or used to support further data mining processes. The latter is the case of 'misleading generalized itemset', a particular type of itemsets obtained from frequent itemsets and a taxonomy of the input data. In this dissertation will be analyzed two real life use cases. The first is related to smart cities while the second will analyze network traffic logs.
\subsection{Dissertation Plan}



\label{Introduction}



\chapter{Frequent Itemset Mining}\label{FIM}
\begin{enumerate}
\item preliminaries and details
\item Why FIM for Big Data
\item Which are the challenges
\end{enumerate}
\chapter{Related works - Survey}
Analysis of the state of the art
\chapter {Frequent Itemset Mining for high dimensional data}
PaMPa-HD
\chapter{Applications of Frequent Itemset Mining to distributed frameworks}
\begin{enumerate}
\item{MGI-Cloud}
\item{Nemico}
\end{enumerate}


\clearpage

%----------------------------------------------------------------------------------------
%	BIBLIOGRAPHY
%----------------------------------------------------------------------------------------

%\addcontentsline{toc}{chapter}{References}
%%%%%% ESWA %%%%%%%%%
\bibliographystyle{IEEEtran}
\bibliography{BiblioESWA1,BiblioESWA2,BiblioICHI,BiblioCBMS,BiblioBikeSharing,BiblioAirPollution,BiblioSTtweets,BiblioPatientFlow}
%%%%%%%%%%%%%%%%%%%
\end{document}  
\grid
\grid
