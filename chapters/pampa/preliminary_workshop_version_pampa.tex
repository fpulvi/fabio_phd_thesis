
% Font 12pt, one column, doublespace, letter-size, 1-inch margin

\documentclass[12pt, onecolumn,letterpaper]{article}
\oddsidemargin  0.0in
\evensidemargin 0.0in
\textwidth      6.5in
\headheight     0.0in
\topmargin      0.0in
\textheight     9.0in



% New commands
\usepackage{graphicx}
\usepackage{times}
\usepackage{amsthm}
\usepackage{amsfonts}
\usepackage{amsmath}
\usepackage{amssymb}




\providecommand{\LT}{\ensuremath{L^3}}
\newcommand{\comment}[1]{}

\newcommand{\ESSENCE}{$\mathcal{E}ssence$}
\newcommand{\ESSENCENEW}{$L^3_{Gen}$}

\providecommand{\LTN}{\ensuremath{L^3_N}}


\begin{document}


%%%%%%%%%%%%%%%%%%%%%%%Versione modificata%%%%%%%%%%%%%%%%%
\section*{Cover Letter}
Dear Prof. T. Palpanas and Z. Wu, \\
Please find enclosed the paper "A Parallel Map-Reduce Algorithm to Efficiently Support Itemset Mining on High Dimensional Data" by Daniele Apiletti, Elena Baralis, Tania
Cerquitelli, Paolo Garza, Pietro Michiardi and myself, which we would like to
submit to Big Data Research.\\

Hereby, I declare that this manuscript is the authors’ original work and has not been published
nor it has been submitted simultaneously elsewhere. Furthermore, all authors have checked the
manuscript and have agreed to the submission.\\

Best regards,\\
Fabio Pulvirenti


\newpage
\section*{Preliminary Workshop Version}

In the following we analyze the differences between this paper, submitted
to Big Data Research, and our own previous work.


\medskip
\noindent
{\bf
[1] D.~Apiletti, E.~Baralis, T.~Cerquitelli, P.~Garza, P.~Michiardi and
F.~Pulvirenti
PaMPa-HD: A Parallel MapReduce-Based Frequent Pattern Miner for High-Dimensional
Data.
{\em IEEE ICDM Workshop on High Dimensional Data Mining (HDM)}, Atlantic City,
  NJ, USA, 2015.
}

\medskip
A very preliminary version of the PaMPa-HD algorithm was initially introduced in
[1].
However, the new journal paper submitted to Big Data Research delivers a set
of major contributions which can be divided in two categories: algorithmic and
experimental.
The former set of additions have been designed to improve the performance of the
algorithm in terms of execution time and communication costs.
The latter, instead, represents an additional contribution in terms of
experiments and comparisons,
in order to broadly validate the quality of the proposed approach.

The major algorithmic additions\footnote{The source code of PaMPa-HD can be downloaded from https://github.com/fabiopulvi/PaMPa-HD} include:
\begin{enumerate}
\item An improved Synchronization task is now included in the Job 3 Reducer
phase (Section 4), making the overall approach much faster and less I/O
intensive.
In the previous version, instead, there was an ad-hoc job, burdening the
execution of the algorithm with additional I/O overhead.
\item The mappers of Job 3, which run a local Carpenter from the input
transposed tables, now directly process the closed itemsets of the previous
iteration. This allowed them to store these itemsets and leverage them to
improve the impact of the local pruning, which aims to prevent the exploration
of useless branches.
\item The impact of the local pruning has been improved and, at the same time, a
decrease of the communication costs of the new version of the algorithm has been
introduced, by sorting the transposed tables based on their position on the
enumeration tree. In the previous implementation, the sorting was not regulated
because of Hadoop Distributed File System (HDFS) chunking. Exploring the tree in
this way is more similar to a centralized exploration, enhancing the impact of
the pruning rule. Additionally, since in the workshop version the tables were
sent to the reducers as soon as they were processed (when the $max\_exp$
threshold is reached), this modification leads to a decreasing in communication
costs because less redundant tables are sent to the reducers.
\item Closed itemsets are not sent to the reducers as soon as they are mined,
but only at the end of the process, hence reducing the communication costs
because itemsets are firstly pruned locally.
\item In order to improve the execution time, a set of strategies related to the
$max\_exp$ parameter are presented (Section 5.2). The ratio behind the
strategies is to increase the $max\_exp$ value along the execution, exploiting
the decreasing size of the tables to mine. The benefits of the strategies are
related to (i) a simpler initial parameter tuning, and (ii) a performance boost
up to almost 20\% in some cases.
%%\item To prevent an increasing load unbalancing among the independent tasks of
%the process, related to the increasing of the $max\_exp$ parameter, the
%synchronization phase is forced after 1 hour of computation. In this way, in the
%worst case, the resources are not completely exploited for at most 1 hour.

\end{enumerate}

The other contributions, which have been included to enrich the experiments
(Section 5), are briefly described in the following:
\begin{enumerate}
\item A whole new extremely-high-dimensional dataset has been introduced, it
describes the occupancy rate of different car lanes of San Francisco bay area
freeways. The dataset is useful to analyze the limit, in terms of number of
transactions of the input dataset, of the high-dimensional design of PaMPa-HD.
\item Two new state-of-the-art distributed algorithms, DistEclat and BigFIM, are
used to compare the performance of PaMPa-HD. The algorithms, which have shown to
outperform Parallel FP-Growth (the unique distributed approach introduced in the
workshop version of the paper) were not available at the time of the original
submission.
\item Section 5.2 is completely new, introducing and analyzing the impact of the
$max\_exp$ strategies.
\item The analysis of the impact of the number of transactions, in Section 5.4
is completely new.
\item A detailed and critical analysis of the communication costs and load
balancing of the algorithm has been introduced in Section 5.6.
\end{enumerate}



\end{document}



