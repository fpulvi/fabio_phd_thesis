
 \textbf{Already pruned redundant stuff}
As already discussed, Frequent itemset mining represents a
very popular data mining technique
used for exploratory analysis.
Its popularity is witnessed by the high number of approaches
and implementations.
The most popular techniques to extract frequent itemsets
from a transactional datasets are Apriori~\cite{Agr94}, Fp-growth~\cite{Han00} and, even if less popular, Eclat~\cite{Zaki97newalgorithms}.
%Apriori~\cite{Agr94} is a bottom up approach:
%itemsets are extended one item at a time and their frequency is tested against
%the dataset.
%FP-growth~\cite{Han00}, instead, is based on an FP-tree transposition of the
%transactional dataset
%and a recursive divide-and-conquer approach.
These techniques explore the search space enumerating the items.
For this reason, they work very well for datasets
with a small number of items per row,
but their running time increases exponentially
with higher row lengths~\cite{Agr94, Zaki97newalgorithms}.
This behavior is directly inherited by their respective distributed implementations:
Parallel FP-growth~\cite{pfpgrowth},~\cite{citeulike:13636750}, BigFIM and DistEclat~\cite{bigfim}.


%These are not the only distributed and parallel implementations of Frequent Itemset miners.
%\cite{qiu2014yafim} introduces another Apriori-based frequent
%itemset miner. The contribution of this work is focused on the candidates
%handling, which are cached in memory between each iteration.
%In~\cite{zhang2015distributed}, a similar breadth-first approach is introduced, but with the exploitation of a matrix-based pruning in order to significantly reduce the amount of candidates. In~\cite{liang2015sequence}, the breadth-first exploration manner is combined
%with the suffix-based candidate generation.\\
%Finally, for the environments requiring very fast response, some sampling-based techniques have been presented~\cite{riondato2015mining},~\cite{gole2015frequent} and~\cite{Wu2015}. These works are characterized by getting a trade-off between execution time and quality of the results. \\
%While the previous works have been designed for use cases characterized by
%datasets with a large amount of transactions,
As largely discussed, Carpenter algorithm~\cite{Zaki_Carpenter}, 
has been specifically designed to extract frequent itemsets
from high-dimensional datasets (in the order of tens of thousands or more attributes). A detailed introduction to the algorithm is presented in section \ref{Carpenter
algorithm}.
The idea of designing a parallel MapReduce algorithm to efficiently support
itemset mining on high dimensional data was first introduced in~\cite{pampa_v1}.
The PaMPa-HD algorithm significantly enhances the algorithm performance proposed in~\cite{pampa_v1}
by providing (i) a more efficent approach to address synchronization phase, reducing the number of MapReduce jobs; (ii) a
more efficient visit of the transposed tables; (iii) and a set of self-tuning strategies to speed up the performances through a dynamic modification of the $max\_exp$ parameter . Furthermore, this work introduces a wider set of experiment to evaluate, on real datasets, the impact of the number of transaction on the performance, but also communication costs and load balancing, very important in a distributed environment.

Specifically, the original algorithm
exploits an additional independent
synchronization job at each iteration. As already described in
Section~\ref{MapReduce Carpenter}, this implementation includes the
synchronization phase in the Mining Job 3. Therefore, the number of MapReduce
jobs (with their related overhead) are strongly reduced.
Additionally, in order to better exploit the pruning rule in the local Carpenter
iteration in each independent task, all the transposed tables are now processed
(not only expanded) in depth-first order. This strategy decreases the
possibility to explore an useless branch of the tree, i.e. a branch whose
results would be completely overwritten by the closed itemsets obtained by
branches older in depth-first fashion. For instance, the performance improvement from the previous version, measured with Breast Cancer Dataset (minsup=6) is from 6\% to 30\% (depending on the number of independent tasks).
%
%In recent years, the availability of Big Data technologies
%allowed the implementation of these techniques in distributed environments
%such as Apache Hadoop~\cite{HDFS},
%based on the MapReduce paradigm~\cite{ArticoloMapReduceGoogle},
%and Apache Spark~\cite{Zaharia_spark}.
%Parallel FP-growth~\cite{pfpgrowth} is the most popular
%distributed closed frequent itemset mining algorithm.
%The main idea is to process more sub-FP-trees in parallel.
%A dataset conversion is required to make all the FP-trees independent.
%A Spark implementation of Parallel FP-growth has been delivered with MLlib
%Library~\cite{citeulike:13636750}. This version extracts all the frequent
%itemsets and not just the closed ones.
%BigFIM and DistEclat~\cite{bigfim} are two recent methods to extract frequent
%itemsets.
%DistEclat represents a distributed implementation of the Eclat
%algorithm~\cite{Zaki97newalgorithms}
%an approach based on equivalence classes (groups of itemsets sharing the same
%prefixes),
%smartly merged to obtain all the candidates.
%BigFIM is a hybrid approach exploiting both the Apriori and Eclat paradigms.
%BigFIM and DistEclat are divided in two phases. In the first one, the approaches
%use respectively an Apriori-like and Eclat-like strategy to mine the itemsets up
%to a fixed k-length. After that, the itemsets are distributed and used as
%prefixes for the longer itemsets. In the last phase, both
%approaches use Eclat to extract all the closed itemsets.
%In addition,~\cite{qiu2014yafim} introduces another Apriori-based frequent
%itemset miner. The contribution of this work is focused on the candidates
%handling, which are cached in memory between each iteration.
%In~\cite{zhang2015distributed}, a similar breadth-first approach is introduced, but with the exploitation of a matrix-based pruning in order to significantly reduce the amount of candidates. In~\cite{liang2015sequence}, the breadth-first exploration manner is combined
%with the suffix-based candidate generation.
%Finally, for the environments requiring very fast response, some sampling-based techniques have been presented~\cite{riondato2015mining},~\cite{gole2015frequent} and~\cite{Wu2015}. These works are characterized by getting a trade-off between execution time and quality of the results. 
%While the previous works have been designed for use cases characterized by
%datasets with a large amount of transactions,
%Carpenter algorithm~\cite{Zaki_Carpenter}, which inspired PaMPa-HD,
%has been specifically designed to extract frequent itemsets
%from high-dimensional datasets, i.e., characterized by a very large number of
%attributes (in the order of tens of thousands or more).
%The basic idea is to investigate the row set space instead of the itemset
%space.
%%A detailed introduction to the algorithm is presented in section \ref{Carpenter
%%algorithm}.
%The idea of designing a parallel MapReduce algorithm to efficiently support
%itemset mining on high dimensional data was first introduced in~\cite{pampa_v1}.
%The PaMPa-HD algorithm significantly enhances the algorithm performance proposed in~\cite{pampa_v1}
%by providing (i) a more efficent approach to address synchronization phase, reducing the number of MapReduce jobs; (ii) a
%more efficient visit of the transposed tables; (iii) and a set of self-tuning strategies to speed up the performances through a dynamic modification of the $max\_exp$ parameter . Furthermore, this work introduces a wider set of experiment to evaluate, on real datasets, the impact of the number of transaction on the performance, but also communication costs and load balancing, very important in a distributed environment.
%
%This work extends our previous work~\cite{pampa_v1}. The original algorithm
%exploits an additional independent
%synchronization job at each iteration. As already described in
%Section\ref{MapReduce Carpenter}, this implementation includes the
%synchronization phase in the Mining Job 3. Therefore, the number of MapReduce
%jobs (with their related overhead) are strongly reduced.
%Additionally, in order to better exploit the pruning rule in the local Carpenter
%iteration in each independent task, all the transposed tables are now processed
%(not only expanded) in depth-first order. This strategy decreases the
%possibility to explore an useless branch of the tree, i.e. a branch whose
%results would be completely overwritten by the closed itemsets obtained by
%branches older in depth-first fashion. For instance, the performance improvement from the previous version, measured with Breast Cancer Dataset (minsup=6) is from 6\% to 30\% (depending on the number of independent tasks).
%% mi sono riferito all'esperimento di scalabilità del precedente paper.



