
\begin{figure}[!t]
%\renewcommand{\arraystretch}{1.3}
%\centerline
{\subfloat[Horizontal representation of $\mathcal{D}$]{
\label{horizontalexampledataset}
\begin{tabular}{|c|l|}
\hline
\multicolumn{2}{|c|}{$\mathcal{D}$}\\
\hline
\hline
	tid & items \\
\hline
	1 & a,b,c,l,o,s,v \\
\hline
	2 & a,d,e,h,l,p,r,v \\
\hline
	3 & a,c,e,h,o,q,t,v \\
\hline
	4 & a,f,v \\
\hline
	5 & a,b,d,f,g,l,q,s,t \\
\hline
\end{tabular}}}%
\hfil
{\subfloat[Transposed representation of $\mathcal{D}$]{
\label{TTexampledataset}
\begin{tabular}{|l|l|}
\hline
\multicolumn{2}{|c|}{$TT$}\\
\hline
\hline
	item & tidlist \\ \hline
	a & 1,2,3,4,5 \\ \hline
	b & 1,5 \\ \hline
	c & 1,3 \\ \hline
	d & 2,5 \\ \hline
	e & 2,3 \\ \hline
	f & 4,5 \\ \hline
	g & 5 \\ \hline
	h & 2,3 \\ \hline
	l & 1,2,5 \\ \hline
	o & 1,3 \\ \hline
	p & 2 \\ \hline
	q & 3,5 \\ \hline
	r & 2 \\ \hline
	s & 1,5 \\ \hline
	t & 3,5 \\ \hline
	v & 1,2,3,4 \\ \hline
\end{tabular}}}%
\hfil
{\subfloat[$TT|_{\{2,3\}}$: example of con\-di\-tio\-nal
tran\-spo\-sed table]{
\label{conditionalexampledataset}
\begin{tabular}{|l|l|}
\hline
\multicolumn{2}{|c|}{$TT|_{\{2,3\}}$}\\
\hline
\hline
item & tidlist \\ \hline
a &4,5 \\ \hline
e & - \\ \hline
h & -\\ \hline
v &4 \\ \hline
\end{tabular}}}%
\caption{Running example dataset $\mathcal{D}$}
\label{exampledataset}
\end{figure}
%
%Let $\mathcal{I}$ be a set of items. A transactional dataset $\mathcal{D}$
%consists of a set of transactions $\{t_1, \dots, t_n\}$,
%where each transaction $t_i\in \mathcal{D}$ is a set of items (i.e.,
%$t_i\subseteq \mathcal{I}$)
%and it is identified by a transaction identifier ($tid_i$).
%Figure~\ref{horizontalexampledataset} reports an example of a transactional
%dataset with 5 transactions.
%It is used as a
%running example through the paper.
%
%An itemset $I$ is defined as a set of items (i.e., $I\subseteq\mathcal{I}$) and
%it is characterized by a tidlist and a support value.
%The tidlist of an itemset $I$, denoted by $tidlist(I)$, is defined as the set of
%tids of the transactions in $\mathcal{D}$ containing $I$,
%while the support of $I$ in $\mathcal{D}$, denoted by $sup(I)$, is defined as
%the ratio between the number of transactions in $\mathcal{D}$ containing $I$
%and the total number of transactions in $\mathcal{D}$ (i.e.,
%$|tidlist(I)|/|\mathcal{D}|$).
%For instance, the support of the itemset \textit{\{aco\}} in
%the running example dataset $\mathcal{D}$ is 2/5 and its tidlist is $\{1,3\}$.
%An itemset $I$ is considered frequent if its support is greater than a
%user-provided minimum support threshold $minsup$.
%
%Given a transactional dataset $\mathcal{D}$ and a minimum support threshold
%$minsup$, the Frequent Itemset Mining \cite{KumarBook} problem consists in
%extracting the complete set of frequent itemsets from $\mathcal{D}$.
%In this paper, we focus on a valuable subset of frequent itemsets called
%frequent closed itemsets~\cite{Zaki_Carpenter}. Closed itemsets
%allow representing the same information of traditional frequent itemsets in a
%more compact form.
%In addition, an item or itemset $I$ is closed in  $\mathcal{D}$ if there exists no superset that
%has the same support count as  $I$.\\
%For instance, in our running example, given a $minsup=2$, the itemset \textit{\{ab\}} is a frequent itemset (support=2), but it is not closed for the presence of the itemset \textit{\{abls\}} (support=2). 
%% %If an itemset is frequent, all of its subsets are frequent for the
%% \textit{monotonic property}.
%% %For the same rule, if an items or an itemset is not frequent, none of its
%% supersets are frequent. In addition, an items
%% % or itemset $I$ is closed in  $\mathcal{D}$ if there exists no superset that
%% has the same support count as  $I$.\\
Since Frequent Itemset Mining preliminaries were introduced far before in the dissertation,
let us just recall and deepen the key concepts fundamental to better understand PaMPa-HD and its
enumeration tree-based exploration strategy. The running example has been slightly modified from the one presentes in Chapter \ref{background}.

As already mentioned, a transactional dataset can also be represented in a vertical format, which is
usually a more effective representation of the dataset when the average number
of items per transactions is orders of magnitudes larger than the number of
transactions.
This representation, called \textit{transposed table} $TT$, assumes that each
row consists of an item $i$ and its list of transactions, i.e.,
$tidlist(\{i\})$. Let $r$ be an arbitrary row of $TT$, $r.tidlist$ denotes the
tidlist of row $r$.
Figure~\ref{TTexampledataset} reports the transposed representation of the
running example reported in Figure~\ref{horizontalexampledataset}.

Given a transposed table $TT$ and a tidlist $X$, the conditional transposed
table of $TT$ on the tidlist $X$, denoted
by $TT|_{X}$, is defined as a transposed table such that:
(1) for each row $r_i\in TT$ such that $X\subseteq r_i.tidlist$ there exists one
tuple $r_i^{\prime}\in TT|_{X}$ and
(2) $r_i^{\prime}$ contains all tids in $r_i.tidlist$ whose tid is higher than
any tid in $X$.
For example, consider the transposed table $TT$ reported in
Figure~\ref{TTexampledataset}. The
projection of $TT$ on the tidlist \textit{\{2,3\}} is the transposed table
reported in Figure~\ref{conditionalexampledataset}.
Each transposed table $TT|_{X}$ is associated with an itemset composed by the
items in $TT|_{X}$.
For instance, the itemset associated with $TT|_{\{2,3\}}$ is \textit{\{aehv\}}
(see Figure~\ref{conditionalexampledataset}).


% %Given a row or a set of rows (in the sequel, \textit{row set}) $x$, $TT|_{x}$,
% is the projection of the whole vertical datasets on the row set x. Each
% transposed table is associated with an itemset composed of the items in the
% table. For each item, the associated tidlist is composed of the tids greater
% than any tid in the row set x. For instance, $TT|_{2,3}$ in tab 3 is the
% transposed table of the row set \textit{2,3} and it represents the itemset
% \textit{aehv}.

